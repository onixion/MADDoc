% KSN MATURA (WEISS)
% Autoren:
%	Alin Porcic
%	Manpreet Singh
% Daniel Ranalter
%
%--------------------------

\documentclass[a4paper]{report}
\usepackage[utf8]{inputenc}
\usepackage[german]{babel}
\usepackage{graphicx}
\usepackage{geometry}
\geometry{a4paper, left=25mm, right=25mm, top=20mm, bottom=30mm}

\title{Fragen Diplomarbeit \\Porcic, Ranalter, Singh, Stojanovic}
\begin{document}
\maketitle
\chapter{Alin Porcic}
\begin{enumerate}
\item Wie wurde die Kommunikation CLI-Server/Client realisiert?
\item Wie werden die Netzwerkaufgaben vom Programm gestartet?
\item Wie haben Sie die CLI umgesetzt? Wie funktioniert sie?
\item Wie lädt das Programm die Konfiguration? Hätte es andere Möglichkeiten gegeben?
\end{enumerate}

\chapter{Daniel Ranalter}
\begin{enumerate}
\item Was ist der Unterschied zwischen den MACFinder Versionen?
\item Wie funktioniert die Implementation der SNMP Traffic Abfrage?
\item Warum musste für die Dritte Version des MACFinders die externe Bibliothek LibPcap?
\item Wie funktionieren die implementierten Jobs zur Überprüfung der Erreichbarkeit von FTP, DNS und SNMP?
\end{enumerate}

\chapter{Manpreet Singh}
\begin{enumerate}
\item Was sind die Unterschiede zwischen der 1.Version und der 2.Version des Notification Systems ?
\item Wieso brauchte es 3 Versionen bis zur finalen Version des GUIs ?
\item Wieso bietet das Notification System so viele Einstellungsmölichkeiten an ?
\item Wie reagiert das Notification System auf Fehlversuche ?
\end{enumerate}

\chapter{Marko Stojanovic}
\begin{enumerate}
\item Wie ist die Datenbank aufgebaut, welche Unterschiede gibt es gegenüber den Vorgängerversionen der Datenbankrealisierung?
\item Wie werden Zeitdaten in die Datenbank geschrieben?
\item Wieso gibt es eine Möglichkeit die Datenbank zusammen zu fassen und wie funktioniert diese?
\item Wieso wurde SQLite zur Realisierung der Datenbank gewählt, welche weiteren Möglichkeiten gibt es noch?
\end{enumerate}

\end{document}
