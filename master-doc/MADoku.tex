\documentclass[10pt,a4paper]{report}
\usepackage[utf8]{inputenc}
\usepackage[german]{babel}
\usepackage{amsmath}
\usepackage{amsfonts}
\usepackage{amssymb}
\usepackage{graphicx}
\usepackage{fancyhdr}
\usepackage{geometry}
\geometry{a4paper, left=25mm, right=25mm, top=20mm, bottom=30mm}


\title{MAD-Network Monitoring\\
Diplomarbeit 2014/15}
\author{Porcic Alin, Ranalter Daniel, Singh Manpreet\\
Betreuer: Dr. Michael Weiss\\
Höhere Technische Bundes Lehr- und Versuchsanstalt Anichstraße\\
Abteilung Höhere Elektronik und Technische Informatik\\
5bHEL}

\begin{document}
\maketitle
\begin{abstract}

\end{abstract}
\setcounter{tocdepth}{6}
\tableofcontents
\newpage
\chapter{Aufgabenstellung}
lorem ipsum
\chapter{Theorie zu den einzelnen Gebieten der Arbeit}
\section{Wahl der Sprache}
lorem ipsum
\section{Kryptologie}
\subsection{Allgemeines}
lorem ipsum
\subsection{Kryprographie}
lorem ipsum
\subsubsection{Geschichte der Kryptographie}
lorem ipsum
\paragraph{Klassische Kryptographie}
lorem ipsum
\paragraph{Moderne Kryptographie}
lorem ipsum
\subsubsection{Ziele der Kryptographie}
lorem ipsum
\subsubsection{Methoden}
lorem ipsum
\subsection{Kryptoanalyse}
lorem ipsum
\subsubsection{Geschichte der Kryptoanalyse}
lorem ispum
\subsubsection{Ziele der Kryptoanalyse}
lorem ipsum
\subsubsection{Methoden}
lorem ipsum
\subsection{Verschlüsselungsverfahren}
lorem ipsum
\subsubsection{Symmetrische Verschlüsselungsverfahren}
lorem ipsum
\paragraph{Merkmale}
lorem ipsum
\paragraph{Nennenswerte symmetrische Verschlüsselungssysteme}
lorem ipsum
\subparagraph{DES}
lorem ipsum
\subparagraph{3DES}
lorem ipsum
\subparagraph{IDEA}
lorem ipsum
\subparagraph{CAST}
lorem ipsum
\subparagraph{RC4}
lorem ipsum
\subparagraph{RC5, RC5a, RC6}
lorem ipsum
\subparagraph{A5}
lorem ipsum
\subparagraph{Blowfish}
lorem ipsum
\subparagraph{Twofish}
lorem ipsum
\subparagraph{AES}
lorem ipsum
\subsubsection{Asymmetrische Verschlüsselungsverfahren}
lorem ipsum
\paragraph{Merkmale}
lorem ipsum
\subparagraph{Digitale Signatur}
lorem ipsum
\subparagraph{Zertifikate}
lorem ipsum
\paragraph{Nennenswerte asymmetrische Verschlüsselungssysteme}
lorem ipsum
\subparagraph{Diffie-Hellman}
lorem ipsum
\subparagraph{RSA}
lorem ispum
\subparagraph{ElGamal}
lorem ipsum
\subsubsection{Hybride Verschlüsselungsverfahren}
lorem ipsum
\paragraph{Merkmale}
lorem ipsum
\paragraph{Nenneswerte hybride Verschlüsselungssysteme}
lorem ipsum
\subparagraph{IPsec}
lorem ipsum
\subparagraph{TLS/SSL}
lorem ipsum
\subparagraph{PGP}
lorem ipsum
\subsubsection{Hash-Verfahren}
lorem ipsum
\paragraph{Merkmale}
lorem ipsum
\paragraph{Nennenswerte Hashsysteme}
lorem ipsum
\subparagraph{MD2, MD4, MD5}
lorem ipsum
\subparagraph{SHA}
lorem ipsum
\subparagraph{RIPEMD}
lorem ipsum
\section{E-Mail}
lorem ipsum
\subsection{Allgemein E-Mail und Notification}
lorem ipsum
\subsubsection{Senden}
lorem ipsum
\paragraph{Graphische Erklärung}
lorem ipsum
\subsubsection{Empfangen}
lorem ipsum
\paragraph{IMAP}
lorem ipsum
\paragraph{POP}
lorem ipsum
\subsection{E-Mail}
lorem ipsum
\subsubsection{Ursprung/Entstehung}
lorem ipsum
\subsubsection{Bedeutung heute}
lorem ipsum
\paragraph{Zukünftig}
lorem ipsum
\subsubsection{Probleme}
lorem ipsum
\paragraph{Kleine Probleme}
lorem ipsum
\paragraph{Große Probleme - Gefahren}
lorem ipsum
\subsubsection{Sicherheit}
lorem ipsum
\paragraph{Versuche}
lorem ipsum
\paragraph{Was kann ich tun?}
lorem ipsum
\section{Oberfläche}
lorem ipsum
\subsection{Allgemein User Interface (UI)}
lorem ipsum
\subsubsection{Geschichte}
lorem ipsum
\subsubsection{UIs}
lorem ipsum 
\subsubsection{Zukünftig}
lorem ipsum 
\subsection{Grahpical User Interface (GUI)}
lorem ipsum
\subsubsection{Bedeutung}
lorem ipsum
\subsubsection{Wichtigkeit}
lorem ipsum
\subsubsection{Vergleich GUI - Command Line Interface (CLI)}
lorem ipsum
\paragraph{Marktführende}
lorem ipsum
\paragraph{Wichtige Operating System GUIs}
lorem ipsum
\subsubsection{Vor- und Nachteile}
lorem ipsum
\subsubsection{Möglichkeiten der Realisierung}
lorem ipsum
\subsubsection{Genauer}
lorem ipsum
\paragraph{Realisierung}
lorem ipsum
\paragraph{Graphikkarte oder Prozessor}
lorem ipsum
\section{Netzwerke}
lorem ipsum
\subsection{Grundlagen}
lorem ipsum
\subsubsection{Der Host}
\paragraph{MAC-Adressen}
lorem ipsum
\paragraph{IP-Adressen}
lorem ipsum
\paragraph{Ports}
lorem ipsum
\subsubsection{Schichtenmodel}
lorem ipsum
\paragraph{OSI}
lorem ipsum
\paragraph{TCP/IP}
lorem ipsum
\subsubsection{Client-Server Verhältnis}
lorem ipsum
\subsection{Protokolle}
lorem ipsum
\subsubsection{Address Resolution Protocol - ARP}
lorem ipsum
\paragraph{Sicherheitsaspekte}
lorem ipsum
\subsubsection{Internet Protocol - IP}
lorem ipsum
\paragraph{IPv4}
lorem ipsum
\paragraph{IPv6}
\subsubsection{User Datagram Protocol - UDP}
lorem ipsum
\subsubsection{Transmission Control Protocol - TCP}
lorem ipsum
\subsubsection{Dynamic Host Configuration Protocol - DHCP}
lorem ipsum
\subsubsection{Domain Name System - DNS}
lorem ipsum
\subsubsection{Internet Control Message Protocol - ICMP}
lorem ipsum 
\paragraph{ICMP Echo Request/Response - Ping}
lorem ipsum
\subsubsection{File Transport Protcol - FTP}
lorem ipsum
\subsubsection{Simple Network Managing Prorocol - SNMP}
lorem ipsum
\paragraph{Management Information Base}
lorem ipsum
\paragraph{SNMPv2/SNMPv2c}
lorem ipsum
\paragraph{SNMPv3}
lorem ispum
\subsubsection{Hypertext Transfer Protcol - HTTP}
lorem ipsum
\subsection{Netzwerksicherheit}
lorem ipsum
\chapter{Programmrealisierung}
lorem ipsum
\section{JobSystem}
lorem ipsum
\section{Notification}
lorem ipsum
\section{Database}
lorem ipsum
\section{Logging}
lorem ipsum
\chapter{Abbildungsverzeichnis}
lorem ipsum
\chapter{Quellenverzeichnis}
lorem ipsum 
\end{document}