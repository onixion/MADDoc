\documentclass[10pt,a4paper]{report}
\usepackage[utf8]{inputenc}
\usepackage[german]{babel}
\usepackage{amsmath}
\usepackage{amsfonts}
\usepackage{amssymb}
\usepackage{graphicx}
\usepackage{fancyhdr}
\usepackage{geometry}
\usepackage{hyperref}
\geometry{a4paper, left=25mm, right=25mm, top=20mm, bottom=30mm}


\title{Doku Richtlinien}
\author{Ranalter Daniel}

\begin{document}

\chapter{Richtlinien für Doku}
\begin{itemize}
\item Wortwiederholungen: Fachbegriffe so oft wie nötig, andere Wörter mit ca. 20 wörtern abstand.
\item Keine Farben
\item Für Hervorhebung: \textbf{Bold} verwenden (kein \textit{italics} oder \underline{underline})
\item Schriftart: (Sans-Serif)
\item keine persönlichen Formulierungen (ich, wir, sie, ..)
\item möglichst hoher sprach standart 
\item Einzugslose Absätze (doppelter Backslash) für zusammenhängende Dinge innerhalb einer Argumentation, bei neuem Gedankengang oä Absatz mit Abstand (zwei x 'Enter') (sorry an mani und stoj i han des falsch verstanden, is koa horizontaler abstand sondern a vertikaler x:)
\item Stoj: Mathematischen dinger Zentriert und mit begin/end equation
\item grafiken zentriert und mit begin/end figure
\item grafiken dürfen nicht den Textfluss unterbrechen
\item grafiken kein rahmen
\item grafiken selbst zeichnen soweit wie möglich, ansonnsten auf die lizenz achten
\item alle grafik formate solang latex sie unterstützt
\item später dann abstände korrigieren (3,5 links 2,5 rechts)
\item bibtex für Quellen von Bildern und Anderem angeben anleitungs link: \url{http://de.wikibooks.org/wiki/LaTeX-Kompendium:_Zitieren_mit_BibTeX}
\item Kopfzeile links: LAN-Monitoring rechts: Kapitel
\item Fußteile links: Autor (vor jedem Kapitel neu definieren) rechts: seite
\item referenzieren: neben zu referenzierendem Kapitel/Section/whatev: (Backslash)label(Geschweifte Klammer auf)xxx:irgendeinNameDerNurEinmalExistiert(Geschweifte Klammer zu)\\
Wobei xxx für folgendes stehen kann:\\
\begin{center}
\begin{tabular}{|c|c|}
In Ref & Formatierungshierarchisches Equivalent\\
\hline
sec & section\\
itm & item\\
chap & chapter\\
part & part\\
p oder para & paragraph\\
lst & listing\\
fig & figure\\
tab & tabular\\
eq & equation\\
s... & sub...
\end{tabular}
\end{center}

zum einfügen der referenz: (Backslash)ref(GeschweifteKlammer auf)ZuvorBestimmterName(Geschweifte Klammer zu) für Kapitel Nr oder das ganze mit pageref für Seite auf welcher die Referenz zu finden ist. (Beispiel in meiner Teil der Doku)
\item für Code: begin/end lstlisting aus usepackage listings
\end{itemize}
\end{document}